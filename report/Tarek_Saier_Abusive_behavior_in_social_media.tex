% THIS IS SIGPROC-SP.TEX - VERSION 3.1
% WORKS WITH V3.2SP OF ACM_PROC_ARTICLE-SP.CLS
% APRIL 2009
%
% It is an example file showing how to use the 'acm_proc_article-sp.cls' V3.2SP
% LaTeX2e document class file for Conference Proceedings submissions.
% ----------------------------------------------------------------------------------------------------------------
% This .tex file (and associated .cls V3.2SP) *DOES NOT* produce:
%       1) The Permission Statement
%       2) The Conference (location) Info information
%       3) The Copyright Line with ACM data
%       4) Page numbering
% ---------------------------------------------------------------------------------------------------------------
% It is an example which *does* use the .bib file (from which the .bbl file
% is produced).
% REMEMBER HOWEVER: After having produced the .bbl file,
% and prior to final submission,
% you need to 'insert'  your .bbl file into your source .tex file so as to provide
% ONE 'self-contained' source file.
%
% Questions regarding SIGS should be sent to
% Adrienne Griscti ---> griscti@acm.org
%
% Questions/suggestions regarding the guidelines, .tex and .cls files, etc. to
% Gerald Murray ---> murray@hq.acm.org
%
% For tracking purposes - this is V3.1SP - APRIL 2009

\documentclass{proseminar}

\begin{document}

\conferenceinfo{Albert-Ludwigs Universit\"at Freiburg\\Technische Fakult\"at, Institut f\"ur Informatik\\Lehrstuhl f\"ur Datenbanken \& Informationssysteme}{}

\title{Abusive behavior in social media}

\numberofauthors{1}
\author{
Tarek Saier\\
\email{tareksaier@gmail.com}
}

\maketitle

\section{Introduction}
Usage of online platforms of all shapes and sizes nowadays is a common part of may people's everyday life. Just like human interaction offline, user interaction on Facebook, Twitter, online forums etc. is not always positive. For all the good like helpful contributions to Wikipedia and engaged discussion on reddit, there also is abusive behaviour taking place.

While the seriousness of the effects such behaviour can have on victims may have been downplayed in earlier days of the web, it is clearly as a serious problem. Furthermore, with the media reporting on large social networks failing to control abusive behaviour and thus influencing their public image, it is in the financial interest of companies running such networks to detect and remove or, if possible, even prevent such behavior.

This report will give an introduction into the topic of \emph{Abusive behavior in social media} --- or more precise: the decection of such behavior --- and is structured as follows. Section 2 will give a wider view on the topic, provide necessary background information and shortly describe approaches for trackling the problem at hand. In section 3 the focus will be put on machine learning as one possible approach. While giving a short overview of the steps of a machine learning procedure in general, noteworthy particularities with regards to abusive behavior in social media will be explained. Section 4 will introduce two concrete approaches --- efforts for detecting abusive comments on Yahoo! on the one hand and agressive Twitter accounts on the other. This will be followed by a comparison of the two. Lastly, section 5 will conclude the report.

\section{Background}
- Wider overview before concentrating on ML afterwards
\subsection{Problem formulation}
- What's the general problem\\
- What are the challenges (language changing, sarcasm, etc.)
\subsection{Approaches for solving the problem}
- Most basic: blacklisting of words\\
- More sophisticated: machine learning, deep learning, etc.

\section{Machine learning}
- \emph{short} description/recap of ML approach / noteworthy particularities with regards to topic at hand
\subsection{Data collection}
- No de facto testing set for abusive language\cite{Yahoo:2016}
\subsection{Feature extraction}
-
\subsection{Learning}
-
\subsection{Evaluation}
-

\section{Two concrete approaches}
-
\subsection{Abusive Yahoo! comments}
- Description and discussion of \cite{Yahoo:2016}\\
\hphantom{- }- NLP features (e.g. \cite{Distributed:2014})\\
\hphantom{- }- ''Vowpal Wabbit's regression model''\\
\hphantom{- }-
\subsection{Aggressive Twitter accounts}
- Description and discussion of \cite{Twitter:2017}\\
\hphantom{- }- WEKA, Random Forest\\
\hphantom{- }-

\subsection{Comparison}
- How do \cite{Yahoo:2016} and \cite{Twitter:2017} compare\\
\hphantom{- }- Classifying accounts (more features) vs. just comments\\
\hphantom{- }- Hate speech, derogatory language, profanity vs.\\
\hphantom{- - }bullying, aggression\\
\hphantom{- }- Ground truth: trained staff vs. crowd sourcing\\
\hphantom{- }- \\
- To what extend are they comparable

\section{Conclusion}
-

% \section{}
% \subsection{}
% \subsubsection{}
% \footnote{}
% \begin{math}\lim_{n\rightarrow \infty}x=0\end{math}
% \begin{equation}\lim_{n\rightarrow \infty}x=0\end{equation}
% \begin{displaymath}\sum_{i=0}^{\infty} x + 1\end{displaymath}
% \begin{table}
% \centering
% \caption{Frequency of Special Characters}
% \begin{tabular}{|c|c|l|} \hline
% Non-English or Math&Frequency&Comments\\ \hline
% \O & 1 in 1,000& For Swedish names\\ \hline
% $\pi$ & 1 in 5& Common in math\\ \hline
% \$ & 4 in 5 & Used in business\\ \hline
% $\Psi^2_1$ & 1 in 40,000& Unexplained usage\\
% \hline\end{tabular}
% \end{table}
%
% \begin{table*}
% \centering
% \caption{Some Typical Commands}
% \begin{tabular}{|c|c|l|} \hline
% Command&A Number&Comments\\ \hline
% \texttt{{\char'134}alignauthor} & 100& Author alignment\\ \hline
% \texttt{{\char'134}numberofauthors}& 200& Author enumeration\\ \hline
% \texttt{{\char'134}table}& 300 & For tables\\ \hline
% \texttt{{\char'134}table*}& 400& For wider tables\\ \hline\end{tabular}
% \end{table*}
%
% \begin{figure}
% \centering
% \epsfig{file=fly.eps}
% \caption{A sample black and white graphic (.eps format).}
% \end{figure}
%
% \newtheorem{theorem}{Theorem}
% \begin{theorem}
% Let $f$ be continuous on $[a,b]$.  If $G$ is
% an antiderivative for $f$ on $[a,b]$, then
% \begin{displaymath}\int^b_af(t)dt = G(b) - G(a).\end{displaymath}
% \end{theorem}
%
% \begin{figure*}
% \centering
% \epsfig{file=flies.eps}
% \caption{A sample black and white graphic (.eps format)
% that needs to span two columns of text.}
% \end{figure*}

\bibliographystyle{abbrv}
\bibliography{bibliography}  % bibliography.bib is the name of the Bibliography in this case

% remember to run:
% latex bibtex latex latex
% to resolve all references
%
% ACM needs 'a single self-contained file'!

\balancecolumns
\end{document}
