% THIS IS SIGPROC-SP.TEX - VERSION 3.1
% WORKS WITH V3.2SP OF ACM_PROC_ARTICLE-SP.CLS
% APRIL 2009
%
% It is an example file showing how to use the 'acm_proc_article-sp.cls' V3.2SP
% LaTeX2e document class file for Conference Proceedings submissions.
% ----------------------------------------------------------------------------------------------------------------
% This .tex file (and associated .cls V3.2SP) *DOES NOT* produce:
%       1) The Permission Statement
%       2) The Conference (location) Info information
%       3) The Copyright Line with ACM data
%       4) Page numbering
% ---------------------------------------------------------------------------------------------------------------
% It is an example which *does* use the .bib file (from which the .bbl file
% is produced).
% REMEMBER HOWEVER: After having produced the .bbl file,
% and prior to final submission,
% you need to 'insert'  your .bbl file into your source .tex file so as to provide
% ONE 'self-contained' source file.
%
% Questions regarding SIGS should be sent to
% Adrienne Griscti ---> griscti@acm.org
%
% Questions/suggestions regarding the guidelines, .tex and .cls files, etc. to
% Gerald Murray ---> murray@hq.acm.org
%
% For tracking purposes - this is V3.1SP - APRIL 2009

\documentclass{proseminar}
\usepackage[hidelinks]{hyperref}

\begin{document}

\conferenceinfo{Albert-Ludwigs Universit\"at Freiburg\\Technische Fakult\"at, Institut f\"ur Informatik\\Lehrstuhl f\"ur Datenbanken \& Informationssysteme}{}

\title{Abusive behavior in social media}

\numberofauthors{1}
\author{
Tarek Saier\\
\email{tareksaier@gmail.com}
}

\maketitle

\section{Introduction}
Usage of online platforms of all shapes and sizes nowadays is a common part of may people's everyday life. Just like human interaction offline, user interaction on Facebook, Twitter, online forums etc. is not always positive. For all the good like helpful contributions to Wikipedia and engaged discussion on reddit, there also is abusive behavior taking place.

While the seriousness of the effects such behavior can have on victims may have been downplayed in earlier days of the web, it is clearly as a serious problem. Furthermore, with the media reporting on large social networks failing to control abusive behavior and thus influencing their public image, it is in the financial interest of companies running such networks to detect and remove or, if possible, even prevent such behavior.

This report will give an introduction into the topic of \emph{Abusive behavior in social media} --- or more precise: the decection of such behavior --- and is structured as follows. Section 2 will give a wider view on the topic, provide necessary background information and shortly describe approaches for trackling the problem at hand. In section 3 the focus will be put on machine learning as one possible approach. While giving a short overview of the steps of a machine learning procedure in general, noteworthy particularities with regards to abusive behavior in social media will be explained. Section 4 will introduce two concrete approaches --- efforts for detecting abusive comments on Yahoo! on the one hand and agressive Twitter accounts on the other. This will be followed by a comparison of the two. Lastly, section 5 will conclude the report.

\section{Background}
In the physical world, abusive behavior can take many forms. Acts of bullying, for example, can be categorized into four types: physical, verbal, relational and damage to property\cite{bullying:2014}. In social media, physical abuse is not possible and damage to property at least rather unlikely and certainly not commonplace. While relational bullying is a possiblity, this report will focus on verbal types of abusive behavior in social media from hereon. Put simply, the remainder of this report is concerned with detecting absusive or malicious intent in text based communication.

\subsection{Problem formulation}
On a high level of abstraction, the task at hand is detecting and stopping abusive behavior in a social media setting. Within the scope of this report \emph{''behavior''} boils down to communication in text form. This communication may be associated with user accounts. User accounts in turn also may have different qualities of interest.

Looking at it from the perspective of an entity operating a social platform, the problem can be formulated as: given all the information about the actors on our platform and the communication they engage in, how can abusive communication be detected? Depending on what constitues an actor on a platform, different possibilities for approaching that goal exist. For example, in the case of platform that wants its user accounts to be as representative of the real person controlling the account as possible (e.g. Facebook), it might be viable to detect abusive \emph{accounts} in order to stop abusive communication. On the other hand, for a platform where the notion of an account does not hold much informational value, it might be more feasable to try and detect abusive communication from its contents only. Examples for the latter setting might be online comment sections that allow anonymous posting, or platforms like 2channel\footnote{\url{http://2ch.net/}} or 4chan\footnote{\url{https://www.4chan.org/}} that just give each participant an ID, so messages of the same origin can be identified as such, while connecting from a different IP address results in a different ID and the traceability of a common origin of messages is lost.

\subsection{Challenges}
As just described, the amount of information available can pose a challenge for detecting abusive behavior and render certain approaches nonviable. Aside from the mentioned feasablility to model user accounts, brevity of communication (e.g. on Twitter due to its 140 character limit) can also pose a hurdle.

Another set of challanges is given by the fact that the communication to be examined happens in natural language from the hands of humans. This means for example that:
\begin{itemize}
\item Offensive language may intentionally be obfuscated (e.g. \emph{ni9 9er}) rendering simple keyword matching ineffective.
\item Some language might be acceptable within one group of people but offensive within another.
\item The offensive nature of an utterance might only come to light when considering a larger context (e.g. multiple sentences) while its parts taken out of context are harmless.
\item Sarcasm might falsely be detected as abusive language, while constant sarcasm towards a user could also be a form of bullying.
\item Language changing over time might require detection methods to be adapted over and over again.
\end{itemize}

Lastly, advancing the field of abusive behavior detection can be challenging because a lot of work may target different types of abusive behavior (e.g. bullying, aggression, hate speech, derogatory language, profanity) and therefore be incompatible.


\subsection{Approaches for solving the problem}
The following will give a very brief overview of some approaches to detect abusive language, while the remainder of the report will focus on machine learning as one such approach.

\subsubsection{Most basic} An overly simplistic approach --- nevertheless often seen applied in online forums --- is to maintain a list of words that are deemed offensive, and filtering out those words or messages containing at least one blacklisted word. As described before this is easily circumvented by users that intentionally misspell or obfuscate offensive terms. Another problem with this approach is that a lot of harmless words containing offensive terms (e.g. to snigger) might falsely be flagged.

\subsubsection{More sophisticated} More sophisticated approaches don't just take into account the words \emph{as is}, but bring them into context. This can, for example, be done by considering the TFIDF scores of selected words. Furthermore, to allow for variations and obfuscated versions of words, not the words themselves but n-grams are used\cite{Yin:2009}. Recent techniques most often are machine learning\cite{Yahoo:2016}\cite{Twitter:2017} or deep learning\cite{DeepLearning:2017} approaches that take into account many different language-based, network-based and, if appilcable, user-based features.

\section{Machine learning}
To develop a supervised machine learning approach for the task of abusive behavior detection one needs to:
\begin{itemize}
\item have or produce a data set with labels (e.g. \emph{abusive} and \emph{not abusive} --- or more fine-grained)
\item decide on which features to extract from the data
\item decide on a learning algorithm
\item evaluate the system
\end{itemize}
The following sections will briefly describe each step and, if appilcable, highlight noteworthy particularities with regards to abusive behavior in social media.

\subsection{Data collection}
For the algorithm to work on, labeled data is necessary. This can be freely available and established testing sets (like the MovieLens data set\footnote{\url{https://grouplens.org/datasets/movielens/}} for the movie domain) or newly obtained data labeled via crowdsourcing or trained personnel.

In the case of abusive behavior in social media, data collection is still problematic to a certain extend, since there is no de facto testing set for abusive language\cite{Yahoo:2016}.

\subsection{Feature extraction}
The extraction of features from the data is essential in that it dictates what the learning algorithm gets as its input. This step might include \emph{feature processing} --- for example the generation of n-grams from whole words.

\subsection{Learning}
Training

\subsection{Evaluation}
Prediction

\section{Two concrete approaches}
-
\subsection{Abusive Yahoo! comments}
- Description and discussion of \cite{Yahoo:2016}\\
\hphantom{- }- NLP features (e.g. \cite{Distributed:2014})\\
\hphantom{- }- ''Vowpal Wabbit's regression model''\\
\hphantom{- }- Mentions outperforming ''state of the art deep learning approach'' but never makes any referecnce to one
\subsection{Aggressive Twitter accounts}
- Description and discussion of \cite{Twitter:2017}\\
\hphantom{- }- WEKA, Random Forest\\
\hphantom{- }-

\subsection{Comparison}
- How do \cite{Yahoo:2016} and \cite{Twitter:2017} compare\\
\hphantom{- }- Classifying accounts (more features) vs. just comments\\
\hphantom{- }- Hate speech, derogatory language, profanity vs.\\
\hphantom{- - }bullying, aggression\\
\hphantom{- }- Ground truth: trained staff vs. crowd sourcing\\
\hphantom{- }- \\
- To what extend are they comparable

\section{Conclusion}
-

% \section{}
% \subsection{}
% \subsubsection{}
% \footnote{}
% \begin{math}\lim_{n\rightarrow \infty}x=0\end{math}
% \begin{equation}\lim_{n\rightarrow \infty}x=0\end{equation}
% \begin{displaymath}\sum_{i=0}^{\infty} x + 1\end{displaymath}
% \begin{table}
% \centering
% \caption{Frequency of Special Characters}
% \begin{tabular}{|c|c|l|} \hline
% Non-English or Math&Frequency&Comments\\ \hline
% \O & 1 in 1,000& For Swedish names\\ \hline
% $\pi$ & 1 in 5& Common in math\\ \hline
% \$ & 4 in 5 & Used in business\\ \hline
% $\Psi^2_1$ & 1 in 40,000& Unexplained usage\\
% \hline\end{tabular}
% \end{table}
%
% \begin{table*}
% \centering
% \caption{Some Typical Commands}
% \begin{tabular}{|c|c|l|} \hline
% Command&A Number&Comments\\ \hline
% \texttt{{\char'134}alignauthor} & 100& Author alignment\\ \hline
% \texttt{{\char'134}numberofauthors}& 200& Author enumeration\\ \hline
% \texttt{{\char'134}table}& 300 & For tables\\ \hline
% \texttt{{\char'134}table*}& 400& For wider tables\\ \hline\end{tabular}
% \end{table*}
%
% \begin{figure}
% \centering
% \epsfig{file=fly.eps}
% \caption{A sample black and white graphic (.eps format).}
% \end{figure}
%
% \newtheorem{theorem}{Theorem}
% \begin{theorem}
% Let $f$ be continuous on $[a,b]$.  If $G$ is
% an antiderivative for $f$ on $[a,b]$, then
% \begin{displaymath}\int^b_af(t)dt = G(b) - G(a).\end{displaymath}
% \end{theorem}
%
% \begin{figure*}
% \centering
% \epsfig{file=flies.eps}
% \caption{A sample black and white graphic (.eps format)
% that needs to span two columns of text.}
% \end{figure*}

\bibliographystyle{abbrv}
\bibliography{bibliography}  % bibliography.bib is the name of the Bibliography in this case

% remember to run:
% latex bibtex latex latex
% to resolve all references
%
% ACM needs 'a single self-contained file'!

\balancecolumns
\end{document}
